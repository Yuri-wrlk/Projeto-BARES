\subparagraph*{Version 1.\+0.\+0 -\/ Monday May 16th 2016}

by Thiago Cesar Morais Diniz de Lucena \& Yuri Reinaldo da Silva

\subsection*{Introduction }

B\+A\+R\+ES, or {\itshape Basic A\+Rithmetic \hyperlink{class_expression}{Expression} Evaluator based on Stacks}, is a program made in C++ with the main purpose to take a series of arithmetic expressions, analyse it searching if there is any problem within the formula and, if it is everything correct, give the expression result.

The current version does not support algebraic or any other type of numeral other than integers.

\subsection*{Instalation and running }

To install and execute the program, follow these steps\+:


\begin{DoxyEnumerate}
\item The file must be extracted from the accompanying zip folder \char`\"{}\+Project\+\_\+\+Bares.\+zip\char`\"{}, it should come with all the files necessary.
\item Open your terminal and navigate to the folder where the files where extracted to.
\item Simply input the {\bfseries make} command. ..$\ast$ Instead of make, another way to do it is by inputting the following command at the terminal\+:
\end{DoxyEnumerate}

g++ -\/std=c++11 -\/I include \hyperlink{drive__bares_8cpp}{src/drive\+\_\+bares.\+cpp} \hyperlink{functions_8cpp}{src/functions.\+cpp} -\/o bin/drive\+\_\+bares

Then, to execute the program, you just run it by inputting the following command\+: \begin{DoxyVerb}bin/drive\_bares *input file* *output file*
\end{DoxyVerb}


{\itshape input file} should be substituted by the file containing the expressions who should be analyzed.

{\itshape output file} is where you want the results to be saved. It is optional and if you do not wish to designate it the program will automatically generate a file at the \char`\"{}data\char`\"{} folder called \char`\"{}results.\+txt\char`\"{}.

A reminder\+: the input file must contain only expressions with integers. Any unsupported symbol such as a letter will be interpreted as an invalid character and will be properly treated as an error.

\subsection*{Operations and terms supported }

The current version of B\+A\+R\+ES supports the use of\+:


\begin{DoxyItemize}
\item Integer numeric constants (-\/32.\+768 to 32.\+767)
\item Operators (+, −, /, ∗,ˆ, \%)
\item Parenthesis
\end{DoxyItemize}

The operations precedence follows the rules of the table below.

\tabulinesep=1mm
\begin{longtabu} spread 0pt [c]{*2{|X[-1]}|}
\hline
\rowcolor{\tableheadbgcolor}\PBS\centering {\bf Operator }&\PBS\centering {\bf Precedence  }\\\cline{1-2}
\endfirsthead
\hline
\endfoot
\hline
\rowcolor{\tableheadbgcolor}\PBS\centering {\bf Operator }&\PBS\centering {\bf Precedence  }\\\cline{1-2}
\endhead
\PBS\centering () &\PBS\centering 0 \\\cline{1-2}
\PBS\centering +, -\/ &\PBS\centering 1 \\\cline{1-2}
\PBS\centering $\ast$, /, \% &\PBS\centering 2 \\\cline{1-2}
\PBS\centering $^\wedge$ &\PBS\centering 3 \\\cline{1-2}
\PBS\centering -\/ (unary) &\PBS\centering 4 \\\cline{1-2}
\end{longtabu}


\subsection*{Table of possible errors found by B\+A\+R\+ES }

If any expression has an error in it, the program will send the first error\textquotesingle{}s code to the output file and, if aplicable, in which position of the original expression it can be found. The possible errors are\+:

\tabulinesep=1mm
\begin{longtabu} spread 0pt [c]{*2{|X[-1]}|}
\hline
\rowcolor{\tableheadbgcolor}\PBS\centering {\bf Code }&\PBS\centering {\bf Associated error  }\\\cline{1-2}
\endfirsthead
\hline
\endfoot
\hline
\rowcolor{\tableheadbgcolor}\PBS\centering {\bf Code }&\PBS\centering {\bf Associated error  }\\\cline{1-2}
\endhead
\PBS\centering E1 &\PBS\centering A number in the expression is above the integer limit (-\/+ 32.\+767 \\\cline{1-2}
\PBS\centering E2 &\PBS\centering A binary operator does not find it\textquotesingle{}s 2nd operand \\\cline{1-2}
\PBS\centering E3 &\PBS\centering There is a symbol who\textquotesingle{}s neither a supported operator nor a number \\\cline{1-2}
\PBS\centering E4 &\PBS\centering A \textquotesingle{}lost\textquotesingle{} symbol is found within the expression \\\cline{1-2}
\PBS\centering E5 &\PBS\centering There is a \textquotesingle{})\textquotesingle{} without a corresponding \textquotesingle{}(\textquotesingle{} in the expression \\\cline{1-2}
\PBS\centering E6 &\PBS\centering A binary operator does not find it\textquotesingle{}s first operand \\\cline{1-2}
\PBS\centering E7 &\PBS\centering There is a \textquotesingle{}(\textquotesingle{} which does not find a \textquotesingle{})\textquotesingle{} \\\cline{1-2}
\PBS\centering E8$\ast$ &\PBS\centering At some point, the expression leads to a division by zero \\\cline{1-2}
\PBS\centering E9$\ast$ &\PBS\centering At some point, a number higher than the limit is generated \\\cline{1-2}
\end{longtabu}
Obs.\+: The number of the column indicating the error on expressions indicates\+:


\begin{DoxyItemize}
\item The column which a number starts, when the error is on that number (E.\+g.\+: \char`\"{}12 $\ast$ 1800 5\char`\"{} has \char`\"{}\+E4 11\char`\"{} as result.)

$\ast$\+\_\+codes indicated by $\ast$ do not show the column where it ocurred, because they are\+\_\+ {\itshape only found during the expression is being calculated}
\end{DoxyItemize}

\subsection*{Examples of valid and correct expressions }

\tabulinesep=1mm
\begin{longtabu} spread 0pt [c]{*2{|X[-1]}|}
\hline
\rowcolor{\tableheadbgcolor}\PBS\centering {\bf \hyperlink{class_expression}{Expression} }&\PBS\centering {\bf Result  }\\\cline{1-2}
\endfirsthead
\hline
\endfoot
\hline
\rowcolor{\tableheadbgcolor}\PBS\centering {\bf \hyperlink{class_expression}{Expression} }&\PBS\centering {\bf Result  }\\\cline{1-2}
\endhead
\PBS\centering -\/3 -\/ -\/4 &\PBS\centering 1 \\\cline{1-2}
\PBS\centering 35 -\/ 3 $\ast$ (-\/2 + 5)$^\wedge$2 &\PBS\centering 8 \\\cline{1-2}
\PBS\centering 54 / 3 $^\wedge$ (12\%5) $\ast$ 2 &\PBS\centering 12 \\\cline{1-2}
\PBS\centering ((2-\/3)$\ast$ 10 -\/ (2$^\wedge$3$\ast$5)) &\PBS\centering -\/50 \\\cline{1-2}
\PBS\centering -\/3 + 4 &\PBS\centering 1 \\\cline{1-2}
\PBS\centering (27) / (-\/3) &\PBS\centering -\/9 \\\cline{1-2}
\PBS\centering ((((((((((7 $^\wedge$ 2)))))))))) &\PBS\centering 49 \\\cline{1-2}
\end{longtabu}
\subsection*{Bugs and limitations }

Terms non-\/supported by the program\+:


\begin{DoxyItemize}
\item Unary plus. (E.\+g.\+: +3 $\ast$ 5)
\item More than one unary minus in sequence. (E.\+g.\+: --5 $\ast$ -\/-\/-\/-\/---3) 
\end{DoxyItemize}